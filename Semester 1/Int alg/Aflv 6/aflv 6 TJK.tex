\documentclass{article}
\usepackage[a4paper]{geometry}
\usepackage[utf8]{inputenc}
\usepackage[T1]{fontenc} 					% 
\usepackage{lmodern} 						% Skrifttype
\usepackage[danish]{babel}
\usepackage{amsmath,amssymb,bm,mathtools,amsthm}	% Matematik pakker
\usepackage{fancyhdr,lastpage}
\usepackage{subfiles}
\usepackage{graphicx,float}
\usepackage{graphicx}
\usepackage{SASnRdisplay}
\usepackage{mathrsfs}
\setcounter{secnumdepth}{2} 				% No numbering
\fancypagestyle{plain}{
	\fancyhf{}								% Clear header/footer
	\renewcommand{\headrulewidth}{0pt}
	\fancyfoot[C]{Side \thepage \hspace{1pt} of \pageref{LastPage}}
}
\pagestyle{plain}
\usepackage{graphicx}
\usepackage{caption}
\usepackage{subcaption}
\usepackage{appendix}
\usepackage{dsfont}
\usepackage{color}
\usepackage{multicol}
\usepackage{enumerate}
\usepackage{setspace}
\newtheorem{theorem}{Sætning}[section]
\newtheorem{corollary}[theorem]{Korollar}
\newtheorem{lemma}[theorem]{Lemma}
\newtheorem{proposition}[theorem]{Proposition}
\newtheorem{definition}[theorem]{Definition}
\theoremstyle{remark}
\newtheorem{obs}[theorem]{Observation}
\newtheorem{remark}[theorem]{Bemærkning}
\newcommand*{\LargerCdot}{\raisebox{-0.25ex}{\scalebox{1.4}{$\cdot$}}}

\addto\captionsenglish{\renewcommand{\figurename}{Figur}}
\DeclarePairedDelimiter\abs{\lvert}{\rvert}
\numberwithin{equation}{section}

\begin{document}
	\author{Thomas Damgaard Vinther \& Jens Kristian Refsgaard Nielsen}
	\title{Handin 5}
	\date{09-10-18}
	\maketitle
\section{Skyline}
We wish to make an algorithm that produces the silhouette of a city. Each building in the city is represented by a tripple (l,h,r) where l (r) is the coordinate on the horizontal line where the building begins (ends) and h is the height of the building.
\\The sihouette of the city is a vector $(x_0,h_1,x_1,\dots,x_{i-1}.h_i,x_i,\dots,h_n,x_n)$ where $x_j<x_{j+1}$ and $h_j$ represents the height between point $x_{j-1}$ and $x_j$.
\subsection{a)}
Given a silhouette
\begin{equation}
	(x_0,h_1,x_1,\dots,x_{i-1}.h_i,x_i,\dots,h_n,x_n)
\end{equation}
We wish to add a building $(l,h,r)$ we use zhe following algorithm\\
\begin{tabular}{l | c | l}
	Time & Line nr & Pseudocode \\ \hline
	1&0&Build-Radix-Tree(S)\\
	1 & 1 & T.root.key = false\\
	m & 2 & for $s_j \in$ S\\
	\indent $n_j+1$ & 3 &\indent InsertRT($s_j$,T,T.root)\\
\end{tabular}\\\\
\subsection{b)}
Given 2 silhouettes
\begin{align}
	X&=(x_0,h_1,x_1,\dots,x_{i-1}.h_i,x_i,\dots\dots,h_n,x_n)\\
	Y&=(y_0,h'_1,y_1,\dots,y_{i-1}.h'_i,y_i,\dots,h'_m,y_m)
\end{align}
We wish to combine them into one. Assume without loss of generality that $m\leq n$\\
\begin{tabular}{l | c | l}
	Time & Line nr & Pseudocode \\ \hline
	&&CombineSilhouette(X,Y)\\
	m & 1 & for k = 1 to k=m\\
	\indent n & 2 & \indent addBuilding(X,$(y_{k-1},h'_k,y_k)$))\\
\end{tabular}\\\\
If addBuilding works, so does CombineSilhouette.
\subsection{c)}
We wish to write the divide and conquer algorithm to take care of this problem.
\end{document}