\documentclass{article}
\usepackage[a4paper]{geometry}
\usepackage[utf8]{inputenc}
\usepackage[T1]{fontenc} 					% 
\usepackage{lmodern} 						% Skrifttype
\usepackage[danish]{babel}
\usepackage{amsmath,amssymb,bm,mathtools,amsthm}	% Matematik pakker
\usepackage{fancyhdr,lastpage}
\usepackage{subfiles}
\usepackage{graphicx,float}
\usepackage{graphicx}
\usepackage{SASnRdisplay}
\usepackage{mathrsfs}
\setcounter{secnumdepth}{2} 				% No numbering
\fancypagestyle{plain}{
	\fancyhf{}								% Clear header/footer
	\renewcommand{\headrulewidth}{0pt}
	\fancyfoot[C]{Side \thepage \hspace{1pt} of \pageref{LastPage}}
}
\pagestyle{plain}
\usepackage{graphicx}
\usepackage{caption}
\usepackage{subcaption}
\usepackage{appendix}
\usepackage{dsfont}
\usepackage{color}
\usepackage{multicol}
\usepackage{enumerate}
\usepackage{setspace}
\newtheorem{theorem}{Sætning}[section]
\newtheorem{corollary}[theorem]{Korollar}
\newtheorem{lemma}[theorem]{Lemma}
\newtheorem{proposition}[theorem]{Proposition}
\newtheorem{definition}[theorem]{Definition}
\theoremstyle{remark}
\newtheorem{obs}[theorem]{Observation}
\newtheorem{remark}[theorem]{Bemærkning}
\newcommand*{\LargerCdot}{\raisebox{-0.25ex}{\scalebox{1.4}{$\cdot$}}}

\addto\captionsenglish{\renewcommand{\figurename}{Figur}}
\DeclarePairedDelimiter\abs{\lvert}{\rvert}
\numberwithin{equation}{section}

\begin{document}
	\author{Thomas Vinther \& Jens Kristian Refsgaard Nielsen}
	\title{Assignment 1}
	\date{04-09-18}
	\maketitle
\section*{}
Assume we have an infinite long sorted sequence of numbers $x_1 < x_2 < x_3 < \cdots < x_{d-1} < x_d < x_{d+1} < x_{d+2} < \cdots $ and we want to find the position of a number y in this list, i.e. we want to find the index d, such that $y = x_d$ if y is contained in the list, or otherwise if y is not in the list the successor of y in the list, e.g. $x_{d-1} < y <= x_d$ (here we assume $x_0 = -\infty$ if $y < x_1$).
\section*{Subtask 1:}
We construct the following algorithm, given binary search from introduction to algorithms third edition:
\\public int binSearch(x,t,p,r)\{
\\\indent low = p;
\\\indent high = max(p,r+1);
\\\indent while (low < high) \{
\\\indent\indent mid = floor($\frac{low+high}{2}$);
\\\indent\indent if ($x\leq T[mid]$) \{
\\\indent\indent\indent high=mid;
\\\indent\indent\}
\\\indent\indent else \{
\\\indent\indent\indent low = mid+1; 
\\\indent\indent\}
\\\indent \}
\\\indent return high;
\\\}
\\
\\public int infSearch(x,y)\{
\\\indent i = 1;
\\\indent n = max(floor(|y|),1000);
\\\indent if ($y\leq x[n]$)\{
\\\indent\indent return binSearch(y,x,1,n);
\\\indent\{
\\\indent while (true) \{
\\\indent\indent if ($x[in+1] \leq y \leq x[n(i+1)]$) \{
\\\indent\indent\indent return binSearch(y,x,in+1,n(i+1));
\\\indent\indent \}
\\\indent\indent i++;
\\\indent \}
\\\}
\section*{Subtask 2:}
We wish to show that we return the correct index $d$.
\\\\Now there are some special cases worth noting:
\\\textbf{Special case 1:} If the sequence $x$ is bounded and real, it is also convergent. Note that in this case $\lim_{n\rightarrow\infty} x_n = \inf \{k\in\mathbb{R}_+ : \forall n\in\mathbb{N} : x_n \leq k\} = \sup \{x_n : n\in\mathbb{N} \} =: x_\infty$ because it is increasing. Now in this case our algorithm will never terminate if $x_\infty \leq y$. But when confronting Gerth, he said that we should assume that the sequence is unbounded.
\\\textbf{Special case 2:} If $\forall n\in\mathbb{N} : x_n\in\mathbb{N}\setminus\{0\}$ we can simply do a binary search on the subsequence $x_1,\dots,x_{floor(|y|)}$ because in this case $\forall m\in\mathbb{N}:m\leq x_m$, and our algorithm solves this issue before entering the while loop. Due to our smart choice of $n$ in the algorithm.
\\In any other case we divide our sequence into n sized parts, and run binary search on the subsequence $x_{in+1},\dots,x_{n(i+1)}$ for an $i$ such that $x_{in+1}\leq y\leq x_{n(i+1)}$. Such an $i$ exists because the sequence is unbounded. And we know from the lectures that binary search finds the index $d$ that we are tasked to find.
\section*{Subtask 3:}
We wish to determine how many comparisons the algoritm performs.
\begin{align*}
	worstCase(d) = \begin{cases}
	\log_2(n)+c & \text{if } d\leq n \\
	\log_2(n)+roof(\frac{d}{n}) +c & \text{if } d >n
	\end{cases}
\end{align*}
We know that binary search in the worst case makes $\log_2(n)$ comparisons. And we look at $d/n$ intervals before finding the correct interval to binary search.
\\In the best case $y = x_{floor(|y|)/2}$ and we are done in very very few comparisons.n
\end{document}