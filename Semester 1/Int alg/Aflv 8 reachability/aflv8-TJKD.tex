\documentclass{article}
\usepackage[a4paper]{geometry}
\usepackage[utf8]{inputenc}
\usepackage[T1]{fontenc} 					% 
\usepackage{lmodern} 						% Skrifttype
\usepackage[danish]{babel}
\usepackage{amsmath,amssymb,bm,mathtools,amsthm}	% Matematik pakker
\usepackage{fancyhdr,lastpage}
\usepackage{subfiles}
\usepackage{MnSymbol}
\usepackage{graphicx,float}
\usepackage{graphicx}
\usepackage{SASnRdisplay}
\usepackage{mathrsfs}
\setcounter{secnumdepth}{2} 				% No numbering
\fancypagestyle{plain}{
	\fancyhf{}								% Clear header/footer
	\renewcommand{\headrulewidth}{0pt}
	\fancyfoot[C]{Side \thepage \hspace{1pt} of \pageref{LastPage}}
}
\pagestyle{plain}
\usepackage{graphicx}
\usepackage{caption}
\usepackage{subcaption}
\usepackage{appendix}
\usepackage{dsfont}
\usepackage{color}
\usepackage{multicol}
\usepackage{enumerate}
\usepackage{setspace}
\newtheorem{theorem}{Sætning}[section]
\newtheorem{corollary}[theorem]{Korollar}
\newtheorem{lemma}[theorem]{Lemma}
\newtheorem{proposition}[theorem]{Proposition}
\newtheorem{definition}[theorem]{Definition}
\theoremstyle{remark}
\newtheorem{obs}[theorem]{Observation}
\newtheorem{remark}[theorem]{Bemærkning}
\newcommand*{\LargerCdot}{\raisebox{-0.25ex}{\scalebox{1.4}{$\cdot$}}}

\addto\captionsenglish{\renewcommand{\figurename}{Figur}}
\DeclarePairedDelimiter\abs{\lvert}{\rvert}
\numberwithin{equation}{section}

\begin{document}
	\author{Thomas Vinther \& Jens Kristian Refsgaard Nielsen}
	\title{Handin 8}
	\date{19-11-18}
	\maketitle
	\section{Reachability}
	
	Let $G=(V,E)$ be a directed graph in which each vertex $u \in V$ is labeled with a unique integer $L(u)$ from the set $\{1,2,\dots,|V| \}$. For each vertex $u \in V$, let $R(u)=\{v \in V : u \leadsto v \}$ be the set of vertices that are reacheable from $u$. Define $\min(u)$ to be the vertex in $R(U)$ whose label is minimum. Give an $O(V+E)$-time algorithm that computes $\min(u)$ for all vertices $v \in V$.\\

	
\subsection{Solution}	
Consider the following algorithm.\\
\newline\noindent \begin{tabular}{l | c | l}
	Time 			& Line nr & Pseudocode \\ \hline
	$\Theta(V)$ 	& 1 & for $v\in V$\\
	\indent 1 		& 2 & \indent $\min(v)=\mathcal{L}(v) = \infty$\\
	O($V+E$) 		& 3 & SCC($G$)\\
	O($V+E$) 		& 4 & Topological-Sort($G^{SCC}$)\\
	O($V^{SCC}$) 	& 5 & for $v\in V^{SCC}$ in reverse Topological order\\
	\indent O($V^{SCC}$) & 6 & \indent $m = \min\{\mathcal{L}(z) \mid z\in G^{SCC}.Adj(v) \}$, \\ 
					& & \indent \indent  //under convention $\min(\emptyset) = \infty$\\
	\indent 1 		& 7 &\indent if $\mid v\mid>1$ or $(v,v)\in E$\\
	\indent\indent O($V$) & 8 &\indent\indent $\mathcal{L}(v) = \min\big\{ m, \min\{L(u)\mid u\in v \cap V\} \big\} $\\
	\indent 1 		& 9 &\indent  else $\mathcal{L}(v) = m$ \\
	\indent O($V$) 	& 10 & \indent for $u\in v\cap V$\\
	\indent\indent 1 & 11 &\indent\indent $\min(u)=\mathcal{L}(v)$
\end{tabular}\\
\newline\newline\noindent \textbf{Correctness:} Note that this problem has optimal substructure, when we calculate the $\mathcal{L}$ function we need only consider nodes that are greater in the topological sort of the DAG $G^{SCC}$, because these are the only nodes we can communicate with. So if we begin with the maximal element of the top sort and maintain the $\mathcal{L}(v)$ function as an extention of the $L$ function and as the minimal value of all the nodes $v$ communicates with.\\
We define $\min(v)=\infty$ if $v$ does not communicate with any nodes, as this case is not clear from the assignment. \\
We call Strongly-Connected-Components(G)so we can apply Topological-Sort($G^{SCC}$) because $G^{SCC}$ is a Directed-Acyclic-Graph.\\
We make the greedy choice by computing in reverse topological order and saving the smallest $L(v)$ to calculate the communicating vertices' own $L(v)$\\
In line 6 we calculate m, the smallest value $\mathcal{L}(z)$ of nodes that v are adjacent to. Note that the $\mathcal{L}$ values might not yet be defined, in which case we simply use the convention $\min(\emptyset)=\infty$.\\
In line 7-9 we check for loops inside of $v$ in $V$. In the affirmative case we calculate $\mathcal{L}(v)$ to be the minimum value of all nodes that nodes in v communicates to, including themselves because v contains a loop. In the other case we do not include the node in v in the calculation of the minimum, because $v \nleadsto v$ so $v \notin R(v)$.\\
Finally we update the  nodes  in v's min value.\\
\\
\textbf{Runtime:}\\ From the estimates we gave alongside the pseudocode we could think it is a $O\big((V^{SCC})^2\big)$ algorithm. However these estimates are very rough.
Note that we look at each vertex in $E^{SCC}$ exactly once throughout the iterations of the for loop in line 5, and recall that $\sum_{v\in V}|\text{Adj}(v)|=|E|$. In the case that $|E|<|V|$ we get an estimate of $O(V)$ so in total lines 5 and 6 are bounded by $O(V^{SCC}+E^{SCC})$\\
In line 7-11 we note that $\sum_{v\in V^{SCC}}|v|=|V|$, so line 5 and 7 are bounded by $O(V)$, note that $|V^{SCC}|\leq|V|$.\\
In total lines 5 through 11 are bounded by $O(V+V^{SCC}+E^{SCC}) = O(V+E^{SCC})$, and in the grand total the algorithm runs in time
\begin{equation}
	O(V+E)+O(V^{SCC}+E^{SCC})+O(E^{SCC})=O(V+E)
\end{equation}

     
	
\end{document}