\documentclass{article}
\usepackage{amssymb}
\usepackage[a4paper]{geometry}
\usepackage[utf8]{inputenc}
\usepackage[T1]{fontenc} 					% 
\usepackage{lmodern} 						% Skrifttype
\usepackage[danish]{babel}
\usepackage{amsmath,amssymb,bm,mathtools,amsthm}	% Matematik pakker
\usepackage{fancyhdr,lastpage}
\usepackage{subfiles}
\usepackage{graphicx,float}
\usepackage{graphicx}
\usepackage{SASnRdisplay}
\usepackage{mathrsfs}
\setcounter{secnumdepth}{2} 				% No numbering
\fancypagestyle{plain}{
	\fancyhf{}								% Clear header/footer
	\renewcommand{\headrulewidth}{0pt}
	\fancyfoot[C]{Side \thepage \hspace{1pt} of \pageref{LastPage}}
}
\pagestyle{plain}
\usepackage{graphicx}
\usepackage{caption}
\usepackage{subcaption}
\usepackage{appendix}
\usepackage{dsfont}
\usepackage{color}
\usepackage{multicol}
\usepackage{enumerate}
\usepackage{setspace}
\newtheorem{theorem}{Sætning}[section]
\newtheorem{corollary}[theorem]{Korollar}
\newtheorem{lemma}[theorem]{Lemma}
\newtheorem{proposition}[theorem]{Proposition}
\newtheorem{definition}[theorem]{Definition}
\theoremstyle{remark}
\newtheorem{obs}[theorem]{Observation}
\newtheorem{remark}[theorem]{Bemærkning}
\newcommand*{\LargerCdot}{\raisebox{-0.25ex}{\scalebox{1.4}{$\cdot$}}}

\addto\captionsenglish{\renewcommand{\figurename}{Figur}}
\DeclarePairedDelimiter\abs{\lvert}{\rvert}
\numberwithin{equation}{section}

\begin{document}
	\author{Thomas Vinther \& Jens Kristian Refsgaard Nielsen}
	\title{Handin 9}
	\date{28-11-18}
	\maketitle
	\section{Grid graph}
	\textbf{Definition:} We call a graph a grid graph when for some k:
	\begin{align}
		V &= \{ v_{i,j} \mid 1\leq i\leq k,1\leq j\leq k \}\\
		E &= \{ (v_{i,j},v_{i,j+1}) \mid 1\leq i\leq k,1\leq j < k \} \nonumber\\
		&\cup \{ (v_{i,j},v_{i,j-1}) \mid 1\leq i\leq k,1 < j\leq k \} \\
		&\cup \{ (v_{i,j},v_{i+1,j}) \mid 1\leq i < k,1\leq j\leq k \} \nonumber
	\end{align}
	\subsection{a)}
	\noindent Let $n=\mid V\mid$ and $m=\mid E \mid$, we wish to calculate $n$ and $m$ as a function of $k$.\\
	It is fairly obvious that $n(k)=k^2$.\\
	Observe that each of the sets in the union of $E$ contain $k(k-1)$ edges, so in total $m(k) = 3k(k-1) = 3k^2-3k$\\
	\subsection{b)}
	We wish to determine the runtime of Dijkstra$(G,w,s)$ for $s=v_{1,1}$.\\
	Note that if $m = o(n^2/\lg n)$ the runtime is $O(m\lg n)$. Recall that
	\begin{align*}
		o(g(n)) &= \{ f(n \mid \forall c>0 \exists n_0 >0 \forall n\geq n_0 : 0\leq f(n) < cg(n) \}\\
		3k^2-3k &\stackrel{?}{=} o(k^4/\lg(k^2)) = o(k^4/(2\lg(k))
	\end{align*}
	We prove this by induction.\\
	\textbf{Induction start:} k=2\\
	\begin{align*}
		3\cdot 2^2 -3\cdot 2 = 6\\
		2^4/(2\lg(2)) = 2^3 > 6
	\end{align*}
	\textbf{Induction step:} assume the statement holds for k.
	\begin{align}
		3(k+1)^2-3(k+1) &= (3k^2-3k)+6k \stackrel{\text{induction}}{<} k^4/(2\lg(k)) + 6k\label{1}\\
		(k+1)^4/2\lg(k+1) &= k^4/(2\lg(k+1))+\frac{4k^3+6k^2+4k+1}{2\lg(k+1)}\label{2}\\
		2\lg(k+1)6k &\stackrel{\text{intimidation}}{<} 4k^3+6k^2+4k+1\label{3}
	\end{align}
	For large $k$ we have $k^4/\lg(k) \stackrel{\text{asymptotic}}{=} k^4/\lg(k+1)$ and combining (\ref{1}) with (\ref{2}) and (\ref{3}) we have
	\begin{align*}
		3(k+1)^2-3(k+1) &\stackrel{(\ref{1})}{<} k^4/(2\lg(k))+6k \stackrel{\text{asymptotic}}{\leq} k^4/(2\lg(k+1)) +6k\\
		&\stackrel{(\ref{3})}{<} k^4/(2\lg(k+1) + \frac{4k^3+6k^2+4k+1}{2\lg(k+1)}\\
		&\stackrel{(\ref{2})}{=} (k+1)^4/2\lg(k+1)
	\end{align*}
	Completing the induction. So Dijkstra runs in $O(k^2\lg(k))$.
	\subsection{c)}
	We wish to compute $\delta(s,v)$ for all verticies. Consider the following dynamic algorithm. Out of bounds exceptions are treated as $\infty$.\\
	\newline\noindent \begin{tabular}{l | c | l}
		Time & Line nr & Pseudocode \\ \hline
		1 & 1 & $\delta(s,s) = 0$ \\
		$k$ & 2 & for $j=2$ to $k$ \\
		\indent 1 & 3 & \indent $\delta(s,v_{1,j}) = \delta(s,v_{1,j-1})+w(v_{1,j-1},v_{1,j}) $\\
		k & 4 & for $l = 2$ to $k$\\
		\indent k & 5 & \indent for $p = 1$ to $k$\\
		\indent 1 & 6 & \indent\indent $\delta(s,v_{l,p}) = \min\{ \delta(s,v_{l-1,p})+w(v_{l-1,p},v_{l,p}), $\\
		\indent  &  & \indent\indent\indent\indent\indent\indent \text{ }\text{ } $ \delta(s,v_{l,p-1})+w(v_{l,p-1},v_{l,p}) \}$\\
		\indent k & 7 & \indent for $p=0$ to $k-1$\\
		\indent 1 & 8 & \indent\indent $\delta(s,v_{l,k-p}) = \min\{ \delta(s,v_{l,k-(p-1)})+w(v_{l,k-(p-1)},v_{l,k-p}),$\\
		\indent  &  & \indent\indent\indent\indent\indent\indent\indent \text{ } $ \delta(s,v_{l,k-(p)}) \}$\\
		\end{tabular}\\\\\
	This algorithm takes $k^2$ extra space to store the $\delta$ values. And clearly runs in $O(k^2)$.\\
	\textbf{Correctness:} In the bottom row the direct path is shortest due to the assumption that all weights are non negative. In the succeding rows we see that the shortest path to a vertex will come from directly below, to the left or to the right. Line 5 and 6 checks if it is from below or to the left, and lines 7 and 8 implicitely check left, below and right, and every possible entry is checked.
\end{document}