\documentclass[a4paper, 10pt]{article}
\usepackage[utf8]{inputenc}
\usepackage[danish]{babel}

%%% Margins
\usepackage[a4paper,left=1in,right=1in,top=1in,bottom=1in]{geometry}

%%% Title
\title{Aflevering 1}
\author{Studerende 1, 2017xxxxx \and Studerende 2, 2017xxxxx}
\date{X. YYYY 20ZZ}

%%% Code environment
\usepackage{listings}
\usepackage{xcolor}
\definecolor{commentsColor}{rgb}{0.497495, 0.497587, 0.497464}
\definecolor{keywordsColor}{rgb}{0.000000, 0.000000, 0.635294}
\definecolor{stringColor}{rgb}{0.558215, 0.000000, 0.135316}
\lstset{
  basicstyle=\ttfamily\small,                   % the size of the fonts that are used for the code
  breakatwhitespace=false,                      % sets if automatic breaks should only happen at whitespace
  breaklines=true,                              % sets automatic line breaking
  frame=tb,                                     % adds a frame around the code
  commentstyle=\color{commentsColor}\textit,    % comment style
  keywordstyle=\color{keywordsColor}\bfseries,  % keyword style
  stringstyle=\color{stringColor},              % string literal style
  numbers=left,                                 % where to put the line-numbers; possible values are (none, left, right)
  numbersep=5pt,                                % how far the line-numbers are from the code
  numberstyle=\tiny\color{commentsColor},       % the style that is used for the line-numbers
  showstringspaces=false,                       % underline spaces within strings only
  tabsize=2,                                    % sets default tabsize to 2 spaces
  language=Scala
}

%% Document

\begin{document}
\maketitle
\section*{Opgave 9}
\subsection*{Kode}
\begin{lstlisting}
  def eval(e: Exp): Int = e match {
    case IntLit(c) => c
    case BinOpExp(leftexp, op, rightexp) =>
      val leftval = eval(leftexp)
      val rightval = eval(rightexp)
      op match {
        case PlusBinOp() => leftval + rightval
        case MinusBinOp() => ???
        case MultBinOp() => ???
        case DivBinOp() =>
          if (rightval == 0)
            throw new InterpreterError(s"Division by zero", e)
          leftval / rightval
        case ModuloBinOp() => ???
        case MaxBinOp() =>
          if (???) ??? else ???
      }
    case UnOpExp(op, subexp) =>
      val subexpval = eval(subexp)
      op match {
        case NegUnOp() => -subexpval
      }
  }
\end{lstlisting}
\subsection*{Beskrivelse}
Trace-mekanismen fungerer ved ... \\
\\
Når fortolkeren køres med argumenterne \texttt{-run -trace examples/calc1.s} fås ...
\section*{Opgave 10}
\section*{Kode}
\begin{lstlisting}
  def unparse(e: AstNode): String =
    ???
\end{lstlisting}
\subsection*{Beskrivelse}
Unparse-mekanismen fungerer ved ...
\section*{Opgave 11}
\section*{Kode}
\begin{lstlisting}
  /**
    * Expressions.
    */
  sealed abstract class Exp extends AstNode

  case class BinOpExp(leftexp: Exp, op: BinOp, rightexp: Exp) extends Exp

  case class UnOpExp(op: UnOp, exp: Exp) extends Exp

  case class IntLit(c: Int) extends Exp

  /**
    * Binary operators.
    */
  sealed abstract class BinOp

  case class PlusBinOp()extends BinOp

  case class MinusBinOp()extends BinOp

  case class MultBinOp()extends BinOp

  case class DivBinOp()extends BinOp

  case class ModuloBinOp()extends BinOp

  case class MaxBinOp()extends BinOp

  /**
    * Unary operators.
    */
  sealed abstract class UnOp

  case class NegUnOp() extends UnOp
\end{lstlisting}
\section*{Beskrivelse}
I den objekt-orienterede stil ...
\end{document}

%%% Local Variables:
%%% mode: latex
%%% TeX-master: t
%%% End:
