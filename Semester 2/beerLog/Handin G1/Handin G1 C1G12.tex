\documentclass{article}
\usepackage{amssymb}
\usepackage[a4paper]{geometry}
\usepackage[utf8]{inputenc}
\usepackage[T1]{fontenc} 					% 
\usepackage{lmodern} 						% Skrifttype
\usepackage[danish]{babel}
\usepackage{amsmath,amssymb,bm,mathtools,amsthm}	% Matematik pakker
\usepackage{fancyhdr,lastpage}
\usepackage{subfiles}
\usepackage{graphicx,float}
\usepackage{graphicx}
\usepackage{SASnRdisplay}
\usepackage{mathrsfs}
\setcounter{secnumdepth}{2} 				% No numbering
\fancypagestyle{plain}{
	\fancyhf{}								% Clear header/footer
	\renewcommand{\headrulewidth}{0pt}
	\fancyfoot[C]{Side \thepage \hspace{1pt} of \pageref{LastPage}}
}
\pagestyle{plain}
\usepackage{graphicx}
\usepackage{caption}
\usepackage{subcaption}
\usepackage{appendix}
\usepackage{dsfont}
\usepackage{color}
\usepackage{multicol}
\usepackage{enumerate}
\usepackage{setspace}
\newtheorem{theorem}{Sætning}[section]
\newtheorem{corollary}[theorem]{Korollar}
\newtheorem{lemma}[theorem]{Lemma}
\newtheorem{proposition}[theorem]{Proposition}
\newtheorem{definition}[theorem]{Definition}
\theoremstyle{remark}
\newtheorem{obs}[theorem]{Observation}
\newtheorem{remark}[theorem]{Bemærkning}
\newcommand*{\LargerCdot}{\raisebox{-0.25ex}{\scalebox{1.4}{$\cdot$}}}

\addto\captionsenglish{\renewcommand{\figurename}{Figur}}
\DeclarePairedDelimiter\abs{\lvert}{\rvert}
\numberwithin{equation}{section}

\begin{document}
	\author{Morten Fausing \& Jens Kristian Refsgaard Nielsen \& Thomas Vinther }
	\title{Assignment G1}
	\date{07-02-2019}
	\maketitle
	\section*{1.}
	We wish to prove the following fact about pre:  
	\begin{align*}
	\forall x,y \in \Sigma^{*}: pre(xy)= pre(x) \cup \{x\}pre(y) 
	\end{align*}
	
	With pre given as follows: 
	\begin{align*}
	\text{pre}(w):=\{ u\in\Sigma^* \mid \exists v\in\Sigma^* : uv = w \}
	\end{align*}	
	
	We wish to prove the equality between the two sets, and therefore start by showing the inclusion $pre(xy) \subseteq pre(x) \cup \{x\}pre(y) $\\
	Let $z \in pre(xy)$ by definition $ \exists v \in \Sigma^* : zv=xy$\\
	When $z \in pre(xy)$, and by the just mentioned definition, we only have three possible cases for $v$: It is the empty word and $z=xy$, it is a part of y up to being entire y, or it is a part of x (whole x) and entire y. By proving each case, we will have proven the inclusion.
	
	At the same time we define: $ suf(w)=\{u\in \Sigma^* \mid\exists v \in \Sigma^* : vu = w\}$ and divide into the three possible cases: \\
	
	\textbf{Case 1:} $v=\Lambda$ \\
	$ z = xy \in \{x\} pre(xy)$\\
	If v is the empty word, z must be equal to xy, and therefore be in the set of $\{x\}pre(y)$\\
	
	\textbf{Case 2:} $ v \in suf(y) $\\
	$v \in suf(y) \Rightarrow \exists w : wv = y \Rightarrow w\in pre(y)$ which gives us: $z=xw \in \{x\} pre(y)$\\
	If $v \in suf(y)$ which also includes $y$ itself, there must exist a $w$ such that $wv=y$ which in turn implies that $z=xw$ by the definition of $pre(xy)$.\\
	
	\textbf{Case 3:} $v \in suf(x)\{y\}$ \\
	By our definition of $v \in suf(x)\{y\}$ which gives us $\exists w \in \Sigma^* : v = wy $ now we look at $zwy = zv = xy \Rightarrow zw = x \Rightarrow z \in pre(x)$\\
	When $v \in suf(x)\{y\} $ we are able to write $v = wy$, as we have chosen $z \in pre(xy)$ we can once again use the definition and write $zwy=xy$, which implies that $x$ can be written as $zw$ which gives the desired: $x \in pre(x)$ as wanted \\
	
	We have now proved that $pre(xy) \subseteq pre(x) \cup \{x\}pre(y)$, we will now show the other inclusion:\\
	
	$pre(xy) \supseteq  pre(x) \cup \{x\}pre(y) $\\
	Let $z \in pre(x) \Rightarrow \exists v \in \Sigma^* : zv = x \Rightarrow zvy = xy \Rightarrow z \in pre(xy)$\\
	If $z \in pre(x)$  we use the definition, and by adding y we simply enlarge the set and get $z \in pre(xy)$\\
	
	Let $z \in \{x\}pre(y) \Rightarrow \exists v \in \Sigma^* : zv=xy \Rightarrow z \in pre(xy)$\\
	We simply use the definition of pre if $z \in \{x\}pre(y)$\\
	We have now shown $pre(x) \cup \{x\}pre(y) \subseteq pre(xy)$ and proven the desired equality.
	
	
	\section*{2. }
	We define $Pre(L(E)):= \cup_{w\in L(E)} pre(w)$, the prefix language.\\
	We now wish to prove that for each regular expression E there exists another regular expression P such that $Pre(L(E))=L(E)$, i.e. that the set of regular languages is closed under taking prefix languages. We proceed by structural induction. \\
	
	\noindent Basis cases:\\
	If E = $\emptyset$, let P = $\emptyset$:
	\begin{align*}
	\text{Pre}(L(E))=\text{ Pre}(\mathcal{L(\emptyset)}) =\mathcal{L}(\emptyset)=\mathcal{L}(P)
	\end{align*}
	If E = a, let P = a+$\Lambda$:
	\begin{align*}
	\text{Pre}(\mathcal{L}(E))&=\text{Pre}(\mathcal{L}(a)) = \bigcup_{w\in\mathcal{L}(a)}\text{pre}(w) = \text{pre}(a)= \{ \text{a}, \Lambda \} = \mathcal{L}(a + \Lambda)
	\\&=\mathcal{L}(P)
	\end{align*}
	\text{Induction step:}\\
	If E=$E_1+E_2$, the induction hypothesis gives regular expressions $E_i':\text{Pre}(\mathcal{L}(E_i))=\mathcal{L}(E_i')$, choose then the regular expression $P=(E_1'+E_2')$:
	\\Recall that $\Omega: \cup_{d \in (A\cup B)}f(d) = (\cup_{d\in A}f(d)) \cup (\cup_{d\in B}f(d))$
	\begin{align*}
	\text{Pre}(\mathcal{L}(E)) &= \text{Pre} (\mathcal{L}(E_{1} + E_{2}) ) = \text{Pre} 
	(\mathcal{L}(E_{1}) \cup \mathcal{L}(E_{2})) \\
	&= \bigcup_{w\in (\mathcal{L}(E_{1}) \cup \mathcal{L}(E_{2}))} \text{pre(w)}
	\\&\stackrel{\Omega}{=} \left( \bigcup_{w\in \mathcal{L}(E_1)} \text{pre}(w) \right) \cup \left( \bigcup_{w\in \mathcal{L}(E_2)} \text{pre}(w) \right)\\
	&= \text{Pre}(\mathcal{L}(E_{1}))\cup \text{Pre}(\mathcal{L}(E_{2}))\stackrel{I.H.}{=} \mathcal{L}(E_{1}')\cup \mathcal{L}(E_{2}') \\ &= \mathcal{L}(E_1'+E_2') = \mathcal{L}(P)
	\end{align*}
	If E=$E_{1}E_{2}$, the induction hypothesis once again gives regular expressions $E_i':\text{Pre}(\mathcal{L}(E_i))=\mathcal{L}(E_i')$, now we choose the regular expression $P=(E_1'+E_1E_2')$
	\begin{align*}
	\text{Pre} (\mathcal{L}(E)) &= \text{Pre} (\mathcal{L} (E_{1}E_{2}))
	= \text{Pre}(\mathcal{L}(E_{1})\mathcal{L}(E_{2}))\\
	&=\bigcup_{w\in \mathcal{L}(E_1)\mathcal{L}(E_2)} \text{pre}(w)
	\\&= \bigcup_{w_{1}\in \mathcal{L}(E_{1}),w_{2} \in \mathcal{L}(E_{2})} \text{pre}(w_{1}w_{2})\\
	&\stackrel{1.}{=}  \bigcup_{w_{1}\in \mathcal{L}(E_{1}),w_{2} \in \mathcal{L}(E_{2})}( \text{pre} (w_{1}) \cup \{w_{1}\}\text{pre}(w_{2}))\\
	&= \left(\bigcup_{w_{1}\in \mathcal{L}(E_{1}),w_{2} \in \mathcal{L}(E_{2})}\text{pre} (w_{1})\right) \cup \left(\bigcup_{w_{1}\in \mathcal{L}(E_{1}),w_{2} \in \mathcal{L}(E_{2})} \{w_{1}\}\text{pre}(w_{2})) \right) \\
	&= \left(\bigcup_{w_{1}\in \mathcal{L}(E_{1})}\text{pre} (w_{1})\right) \cup \left(\bigcup_{w_{1}\in \mathcal{L}(E_{1})} \{w_{1}\}\right)\left( \bigcup_{w_{2} \in \mathcal{L}(E_{2})} \text{pre}(w_{2})) \right) \\
	&= \text{Pre}(\mathcal{L}(E_{1})) \cup \mathcal{L}(E_{1})\text{Pre}(\mathcal{L}(E_{2}))\\
	&\stackrel{I.H.}{=}\mathcal{L}(E_{1}') \cup \mathcal{L}(E_{1})\mathcal{L}(E_{2}')\\
	&=\mathcal{L}(E_{1}'+E_{1}E_{2}')=\mathcal{L}(P)
	\end{align*}
	
	\noindent Kleene star case:\\ The following will become useful, consider for $i>0$
	\begin{align*}
	\triangledown: \text{Pre}(\mathcal{L}(E)^i) &= \text{Pre}(\mathcal{L}(E)^{i-1}\mathcal{L}(E))\\
	&\stackrel{1)}{=} \text{Pre}(\mathcal{L}(E)^{i-1})\cup\mathcal{L}(E)^{i-1}\cdot\text{Pre}(\mathcal{L}(E))\\
	&=\bigg(\text{Pre}(\mathcal{L}(E)^{i-2})\cup\mathcal{L}(E)^{i-2}\cdot\text{Pre}(\mathcal{L}(E))\bigg)\cup\mathcal{L}(E)^{i-1}\cdot\text{Pre}(\mathcal{L}(E))\\
	&\text{ }\text{ }\vdots\\
	&= \bigcup_{0\leq j < i} \mathcal{L}(E)^j \cdot \text{Pre}(\mathcal{L}(E))\\
	&= \left(\bigcup_{0\leq j < i} \mathcal{L}(E)^j\right) \cdot \text{Pre}(\mathcal{L}(E)) 
	\end{align*}
	By applying part 1 of the assignment a finite number of times or a short induction omitted here, we see that taking powers of prefix languages it is sufficient to concatenate on the right by the prefix language of the base expression.\\
	If $E = E_1^*$, the induction hypothesis grants us a regular expression $E_1': \text{Pre}(\mathcal{L}(E_1)) = E_1'$, choose then the regular expression $P=E_1^*E_1$\\
	\begin{align*}
	\text{Pre}(\mathcal{L}(E))&= \text{Pre}(\mathcal{L}(E_1^*)) = \text{Pre}\left( \bigcup_{i\geq 0} \mathcal{L}(E_1)^i \right)\\
	&=\bigcup_{i\geq 0} \text{Pre}(\mathcal{L}(E_1)^i) \stackrel{\triangledown}{=} \bigcup_{i\geq 0} \mathcal{L}(E_1)^i \text{Pre}(\mathcal{L}(E_1))\\
	&=\mathcal{L}(E_1^*)\text{Pre}(\mathcal{L}(E_1)) \stackrel{I.H.}{=} \mathcal{L}(E_1^*)\mathcal{L}(E_1')\\
	&=\mathcal{L}(E_1^*E_1') = \mathcal{L}(P)
	\end{align*}
	Thus concluding the proof by structural induction.\qed
	\section*{3.}
	By using the previous proof, we extract a function pref on regular expressions, such that\\
	$\mathcal{L}(\text{pref}(E))= \text{Pre}(\mathcal{L}(E))$. Here we show the recursive definition:
	\begin{align*}\label{recursion}
	\text{pref}(E) = \begin{cases}
	\emptyset & \text{if } E= \emptyset\\
	a + \Lambda & \text{if } E=a\\
	\text{pref} (E_1) + \text{pref} (E_2) & \text{if } E=E_1+E_2\\
	\text{pref} (E_1) + E_1 \text{pref} (E_2) & \text{if } E= E_1E_2\\
	E_{1}^* \text{pref} (E_1) & \text{if } E= E_{1}^*
	\end{cases}
	\end{align*}	
	\section*{4.}
	We compute the result of $\text{pref}(a^*(b+cd)^*)$ by using our definition of pref.\\
	\begin{align*}
	\text{pref}(a^*(b+cd)^*)
	&=\text{pref}(a^*)+a^*\text{pref}((b+cd)^*)\\
	&=a^* \text{pref}(a) + a^*(b+cd)^*\text{pref}(b+cd)\\
	&=a^*(a+\Lambda) + a^*(b+cd)^*(\text{pref}(b)+\text{pref}(cd))\\
	&=a^*(a+\Lambda) + a^*(b+cd)^*(b+\Lambda + \text{pref}(c)+(c) \text{pref}(d))\\
	&=a^*(a+\Lambda) + a^*(b+cd)^*(b+\Lambda+(c+\Lambda)+c(d+\Lambda))\\
	\end{align*}
	We can simplify this expression a bit
	\begin{align*}
	\text{pref}(a^*(b+cd)^*) &= a^*(a+\Lambda) + a^*(b+cd)^*(b+\Lambda+(c+\Lambda)+c(d+\Lambda))\\
	&\equiv a^++a^* + a^*(b+cd)^*(b+\Lambda+c+\Lambda+cd+c)\\
	&\equiv a^* + a^*(b+cd)^*(\Lambda+b+c+cd)
	\end{align*}
	In the above we use the equivalence relation on regular expressions:
	\begin{align*}
	E\equiv P \iff \mathcal{L}(E) = \mathcal{L}(P)
	\end{align*}
	Recall that by definition $a^+\subsetneq a^*$ so $\mathcal{L}(a^++a^*)=\mathcal{L}(a^*)$ and $a^++a^*\equiv a^*$.
	
\end{document}