\documentclass{article}
\usepackage{amssymb}
\usepackage[a4paper]{geometry}
\usepackage[utf8]{inputenc}
\usepackage[T1]{fontenc} 					% 
\usepackage{lmodern} 						% Skrifttype
\usepackage[danish]{babel}
\usepackage{amsmath,amssymb,bm,mathtools,amsthm}	% Matematik pakker
\usepackage{fancyhdr,lastpage}
\usepackage{subfiles}
\usepackage{graphicx,float}
\usepackage{graphicx}
\usepackage{SASnRdisplay}
\usepackage{mathrsfs}
\setcounter{secnumdepth}{2} 				% No numbering
\fancypagestyle{plain}{
	\fancyhf{}								% Clear header/footer
	\renewcommand{\headrulewidth}{0pt}
	\fancyfoot[C]{Side \thepage \hspace{1pt} of \pageref{LastPage}}
}
\pagestyle{plain}
\usepackage{graphicx}
\usepackage{caption}
\usepackage{subcaption}
\usepackage{appendix}
\usepackage{dsfont}
\usepackage{color}
\usepackage{multicol}
\usepackage{enumerate}
\usepackage{setspace}
\newtheorem{theorem}{Sætning}[section]
\newtheorem{corollary}[theorem]{Korollar}
\newtheorem{lemma}[theorem]{Lemma}
\newtheorem{proposition}[theorem]{Proposition}
\newtheorem{definition}[theorem]{Definition}
\theoremstyle{remark}
\newtheorem{obs}[theorem]{Observation}
\newtheorem{remark}[theorem]{Bemærkning}
\newcommand*{\LargerCdot}{\raisebox{-0.25ex}{\scalebox{1.4}{$\cdot$}}}

\addto\captionsenglish{\renewcommand{\figurename}{Figur}}
\DeclarePairedDelimiter\abs{\lvert}{\rvert}
\numberwithin{equation}{section}

\begin{document}
	\author{Jens Kristian Refsgaard Nielsen}
	\title{Assignment I1}
	\date{\today}
	\maketitle
\section*{Question 1.}
\textbf{Claim: } the language of the Spanish flag $L = \{ r^n w y^{2n} r^n \mid n\geq 0 \}$ is not context-free\\
\textbf{Proof: } we proceed by using the contraposition of the pumping lemma for context-free languages that states:
\begin{align*}
	&X\subseteq \Sigma^* \text{ is not context free if:}\\
	&\forall k\geq 1 \exists z \in X : |z|>k:\\
	&\forall u,v,\epsilon,x,\zeta\in\Sigma^* : z=uv\epsilon x\zeta, |v\epsilon x|\leq k, |vx|>0:\\
	&\exists i\geq 0 : uv^i\epsilon x^i \zeta \notin X 
\end{align*}
So let $k\geq 1$ be given, we then choose $z=r^k w y^{2k} r^k$, and let a decomposition $u,v,\epsilon,x,\zeta\in\Sigma^* : z=uv\epsilon x\zeta, |v\epsilon x|\leq k, |vx|>0$ be given. Now we proceed by looking at the possible cases of how the decomposition looks.\\
\includegraphics[scale=0.13]{cases}\\
%\textbf{Case 1:} $u = \Lambda, v\epsilon x = r^l, \zeta = r^{k-l}w y^{2k}r^k$ for $0<l\leq k$\\
\\\textbf{Case 1:} $u = r^l, v\epsilon x = r^j, \zeta = r^{k-l-j}w y^{2k}r^k$ for $0\leq l\leq k$ and $0<j<k-l$\\
The condition $|vx|>0$ forces atleast one of $v$ and $x$ to have atleast one $r$, so if we choose to pump with $i = 2$ we get atleast one more $r$ in $v\epsilon x$ so $v^2\epsilon x^2 = r^p$ for $p>j$, consider then
\begin{align*}
	uv^2\epsilon x^2\zeta = r^l r^p r^{k-l-j}w y^{2k}r^k = r^{k-j+p}w y^{2k}r^k \notin L
\end{align*}
Because $p>j$ we get more r's in front of our expression than in the back, which is not allowed in L, concluding the case.\\
\\\textbf{Case 2:} If $w$ is part of either $v$ or $x$, simply pump $i = 2$ and we get two $w$'s which our language doesnt support, completing the small case.\\
\\\textbf{Case 3:} $u = r^{k-l-i}, v = r^l, \epsilon = r^iwy^r, x = y^j, \zeta = y^{2k-j}r^k$ for $0<l+i+1+r+j\leq k$ and $0\leq i<k-l$ and $0\leq r$\\
Note that either $l$ or $j$ is non zero so if we pump $i=2$ we either get too many $r$'s in the front or too many $y$'s compared to the $r$'s in $\zeta$, showing that $uv^2\epsilon x^2 \zeta \notin L$ completing the case.
We have $l$ and $j$ not both zero, if \\
\\\textbf{Case 4:} $u = r^{k}wy^l, v\epsilon x = y^j, \zeta = y^{2k-j-l}r^k$ for $0<j\leq k$ and $0\leq l \leq k$\\
Once again the $|vx|>0$ condition forces either $v$ or $x$ to have atleast one $y$, so when we pump $i=2$ we get more than $2k$ $y$'s in our decomposition and
\begin{align*}
	uv^2\epsilon x^2\zeta = r^k w y^p r^k\stackrel{p>2k}{\notin} L
\end{align*}
\\\textbf{Case 5:} $u = r^{k}wy^{2k-l}, v\epsilon x = y^lr^j, \zeta = r^{k-j}$ for $0<l+j\leq k$\\
If either $v$ or $x$ has both a $y$ and an $r$, we can pump $i=2$ and get something clearly wrong, since we are only allowed to change symbols 3 times throughout the word, but if say $x=y^ar^b$ then $x^2=y^ar^by^ar^b$ for non zero $a$ and $b$, and when we put them into the expression $uv^2\epsilon x^2 \zeta$ we change symbols atleast 4 times, which is not allowed in L. So for the remainder of the case assume $v$ and $x$ only has at most 1 type of letter. Then either $v$ has atleast one $y$ or $x$ has atleast one $r$, so when we pump $i=2$ we have an inconsistency with the number of $r$'s in the front i.e.
\begin{align*}
&uv^2\epsilon x^2\zeta = r^k w y^p r^q, && p\geq 2k, q\geq k
\end{align*}
Where if $v$ has a $y$ we get $p>2k$ or if $x$ has an $r$ we get $q>k$ and since we dont change the front most $r$'s we get that $uv^2\epsilon x^2\zeta\notin L$ as wanted.\\
\\\textbf{Case 6:} $u = r^{k}wy^{2k}r^{k-l-j}, v\epsilon x = r^l, \zeta = r^{j}$ for $0<l$ and $l+j\leq k$\\
Once more unto the breach, the condition $|vs|>0$ saves the day, because then $v$ or $x$ has an $r$, and when we pump $i=2$ we get too many $r$'s in the last batch, compared to the first. \qed\\
\\In conclusion we chose $z = r^kwy^{2k}r^k$ and applied the contraposition of the pumping lemma for context free languages, to show that $L=\{ r^n w y^{2n} r^n \mid n\geq 0 \}\ni z$ is not context free.
\section*{Question 2.}
We wish to construct an unrestricted grammar to describe $L =\{ r^n w y^{2n} r^n \mid n\geq 0 \}$, consider the following
\begin{align}
	&S \rightarrow rRYA\mid w \label{S}\\
	&R \rightarrow w\mid rRY \label{R}\\
	&Y \rightarrow yyZ \label{Y}\\
	&Zyy \rightarrow yyZ \label{Zyy}\\
	&ZA \rightarrow Ar \label{ZA}\\
	&A \rightarrow \Lambda \label{A}
\end{align}
The idea is to start with $S$ and then we can just end with $w$ to represent that word, if however we want an $r$ we need to place 2 $y$'s and finish with another $r$. To accomplish this we put a $Y$ for every $r$ in front, and later use the $Y$ to place 2 $y$'s and a $Z$ that we in turn destroy with $A$ to get the $r$'s in the rear of the word, so the grammar accepts all of $L$. The grammar also doesnt accept any words not in $L$ since the only variables that maps to a terminal without another variable are $S$ and $A$. If we perfom the $S\rightarrow w$ we get the word $w\in L$ so no trouble. If we perfom the derivation $A\rightarrow\Lambda$ early, there will still be variables within the parse that now cannot be resolved and the parseing attempt fails.
\\Consider the following conceptual derivation of $r^nwy^{2n}r^n$
\begin{align*}
	S &\xrightarrow{(\ref{S})} rRYA \xrightarrow{(\ref{R}),n-1} r^nRY^nA
	\\&\xrightarrow{(\ref{Y})} r^nwyyZY^{n-1}A \xrightarrow{(\ref{Y}),n-1} r^nw(yyZ)^{n}A 
	\\&\xrightarrow{(\ref{Zyy})} r^nw(yyZ)^{n-2}yyyyZZA \xrightarrow{(\ref{Zyy}),n(n-1)/2} r^n2y^{2n}Z^nA
	\\&\xrightarrow{(\ref{ZA})} r^n2y^{2n}Z^{n-1}Ar \xrightarrow{(\ref{ZA}),n-1} r^n2y^{2n}Ar^n
	\\&\xrightarrow{(\ref{A})} r^nwy^{2n}r^n
\end{align*}
Note that the $\xrightarrow{(0.4),n(n-1)/2}$ step means applying rule $(0.4)$ on each $Z$ the maximum amount of times possible, the last $Z$ is already in place so we apply this step $\sum_{k=0}^{n-1} k = \frac{n(n-1)}{2}$ times. Note that we should start from the back since we dont have any rules allowing us to move the first of the two $Z$'s in expressions like $yyZZyy$, but of course the sequencencing of the operations is unimportant for the purposes of this exercise, this observation is solely meant for writing an efficient parsing algorithm.
\section*{Question 3.}
\textbf{Claim:} $\forall L_1\in$DCFL, $\forall L_2\in$REG $:L_1\cap L_2 \in$DCFL\\
\textbf{Proof:} Let $L_1\in$DCFL and $L_2\in$REG, then we know that there exists a DPDA say $M=(Q_N,\Sigma_M,\Gamma_M,q_{M,0},A_M,\delta_M)$ and a DFA say $N=(Q_N,\Sigma_N,q_{N,0},A_N,\delta_N)$ such that $\mathcal{L}(M)=L_1$ and $\mathcal{L}(N)=L_2$. We now construct
\begin{align*}
	M\times N = (Q_m\times Q_N,\Sigma_M\cup\Sigma_N,\Gamma_M,(q_{M,0},q_{N,0}),A_M\times A_N,\delta)
\end{align*}
Where $\delta$ is constructed by the following rules
\begin{align}
	&\text{if } q_M \xrightarrow{a\text{ }B/W} q_M' \text{ and } q_N \xrightarrow{a} q_N' \text{ then }&& (q_M,q_N)\xrightarrow{a\text{ }B/W}(q_M',q_N') \label{const1}\\
	&\text{if } q_M \xrightarrow{\Lambda\text{ }B/W} q_M' \text{ then }&& (q_M,q_N)\xrightarrow{a\text{ }B/W}(q_M',q_N) \label{const2}
\end{align}
We wish to prove that $M\times N$ is a DPDA, because then $\mathcal{L}(M\times N) = L_1\cap L_2$ is a Deterministic Context Free Language, as wanted. We know from theorem Lecture 10 slide 6 that the intersection of a CFL and a REG is a CFL, so since DCFL$\subsetneq$CFL the theorem proves that $M\times N$ is a PDA, since we used exactly the same construction as the theorem. So we are left to prove that this construction complies with the added demands of a DPDA, recall that a PDA is a DPDA if the following two criterion are met
\begin{align}
	&\forall q,a,A: |\delta(q,a,A)|\leq 1 \label{DPDA1}\\
	&\text{at most of of }\delta(q,a,A)\text{ and }\delta(q,\Lambda,A)\text{ is non empty} \label{DPDA2}
\end{align}
Assume for contradiction that this is not the case.\\
\textbf{Part 1:} If rule (\ref{DPDA1}) is not true then we must have a tripple
\begin{align*}
	((q_{(M,1)},q_{(N,1)}),a,A) \in (Q_M\times Q_N)\times(\Sigma_M\cup\Sigma_N)\times\Gamma_M:|\delta((q_M,q_N),a,A)|>1
\end{align*}
This means that we have atleast two \textbf{different} pairs in the set, say $((q_{(M,2)},q_{(N,2)}),B)$ and $((q_{(M,3)},q_{(N,3)}),D)$.\\
We further subdivide in the cases $a=\Lambda$ and $a\neq\Lambda$.\\
\\\textbf{Part 1.1:} $a\neq\Lambda$
\\ Now the rules state
\begin{align*}
	&(q_{(M,1)},q_{(N,1)}) \xrightarrow{a\text{ }A/B}(q_{(M,2)},q_{(N,2)}) &&\stackrel{(\ref{const1})}{\iff} q_{(M,1)}\xrightarrow{a\text{ }A/B}q_{(M,2)}\text{ and }q_{(N,1)}\xrightarrow{a}q_{(N,2)}\\
	&(q_{(M,1)},q_{(N,1)}) \xrightarrow{a\text{ }A/D}(q_{(M,3)},q_{(N,3)}) &&\stackrel{(\ref{const1})}{\iff} q_{(M,1)}\xrightarrow{a\text{ }A/D}q_{(M,3)}\text{ and }q_{(N,1)}\xrightarrow{a}q_{(N,3)}
\end{align*}
The implication in the above follows from (\ref{const1}) since we have full knowledge about the way $\delta$ was constructed.\\
Recall that $N$ is a DFA so $q_{(N,1)}\xrightarrow{a}q_{(N,2)}$ and $q_{(N,1)}\xrightarrow{a}q_{(N,3)}$ implies that $q_{(N,2)}=q_{(N,3)}$.\\
Note also that
\begin{align*}
	q_{(M,1)}\xrightarrow{a\text{ }A/B}q_{(M,2)}\iff (q_{(M,2)},B)\in\delta(q_{(M,1)},a,A)\\
	q_{(M,1)}\xrightarrow{a\text{ }A/D}q_{(M,3)}\iff (q_{(M,3)},D)\in\delta(q_{(M,1)},a,A)
\end{align*}
But $M$ is a DPDA, so $|\delta(q_{(M,1)},a,A)|\leq 1$ and then we must have $(q_{(M,3)},D) = (q_{(M,2)},B) \iff q_{(M,2)}=q_{(M,3)}\wedge B=D$.\\
In summary we have shown that
\begin{align*}
	((q_{(M,2)},q_{(N,2)}),a,B)=((q_{(M,3)},q_{(N,3)}),a,D)
\end{align*}
Which is a contradiction since we assumed that the two pairs were different.\\
\\\textbf{Part 1.2:} $a =\Lambda$\\
With $a=\Lambda$ we used (\ref{const2}) to get our pairs, and from the get go $q_{(N,1)}=q_{(N,2)}=q_{(N,3)}$
\begin{align*}
&(q_{(M,1)},q_{(N,1)}) \xrightarrow{\Lambda\text{ }A/B}(q_{(M,2)},q_{(N,1)}) &&\stackrel{(\ref{const2})}{\iff} q_{(M,1)}\xrightarrow{a\text{ }A/B}q_{(M,2)}\\
&(q_{(M,1)},q_{(N,1)}) \xrightarrow{\Lambda\text{ }A/D}(q_{(M,3)},q_{(N,1)}) &&\stackrel{(\ref{const2})}{\iff} q_{(M,1)}\xrightarrow{a\text{ }A/D}q_{(M,3)}
\end{align*}
Now a very similar argument to that of case 1.1 shows that $q_{M,2}=q_{M,3}$, due to (\ref{DPDA1}) and we obtain the contradiction that the two pairs are not different.\\
\\\textbf{Part 2:} In part 1 we have seen that $M\times N$ satisfies the first DPDA criteria (\ref{DPDA1}), now we move on to show the second criteria (\ref{DPDA2}), we continue to use contradiciton.
\\If (\ref{DPDA2}) does not hold we must have a tripple $((q_{(M,1)},q_{(N,1)}),a,A)$ such that both $\delta((q_{(M,1)},q_{(N,1)}),a,A)$ and $\delta((q_{(M,1)},q_{(N,1)}),\Lambda,A)$ are non empty. We then take elements $((q_{(M,2)},q_{(N,2)}),B)\in\delta((q_{(M,1)},q_{(N,1)}),a,A)$ and $((q_{(M,3)},q_{(N,3)}),D)\in\delta((q_{(M,1)},q_{(N,1)}),\Lambda,A)$, schematically this is of course equivivalent to
\begin{align*}
	&(q_{(M,1)},q_{(N,1)}) \xrightarrow{a\text{ }A/B}(q_{(M,2)},q_{(N,2)}) &&\stackrel{(\ref{const1})}{\iff} q_{(M,1)}\xrightarrow{a\text{ }A/B}q_{(M,2)}\text{ and }q_{(N,2)}\xrightarrow{a}q_{(N,2)}\\
	&(q_{(M,1)},q_{(N,1)}) \xrightarrow{\Lambda\text{ }A/D}(q_{(M,3)},q_{(N,3)}) &&\stackrel{(\ref{const2})}{\iff} q_{(M,1)}\xrightarrow{a\text{ }A/D}q_{(M,3)}\text{ and }q_{(N,1)}=q_{(N,3)}
\end{align*}
We obtain our contradiction entirely from the fact that $M$ is a DPDA, because the situation described above exactly means that both $\delta_M(q_{(M,1)},a,A)$ and $\delta_M(q_{(M,1)},\Lambda,A)$ each contain an element, which is not allowed for DPDA's by (\ref{DPDA2}).\qed
\section*{Question 4.}
Given that DCFL is closed under complements we wish to prove that for all $L_1\in$DCFL and $L_2\in$REG we have $L_1\cup L_2\in$DCFL.\\
\\\textbf{Proof:}\\
For $X\subseteq\Sigma^*$ denote by $X^c$ the complement $\Sigma^*\setminus L$ and consider
\begin{align}
	&(L_1\cup L_2)^c \stackrel{\text{De Moivre}}{=} L_1^c\cap L_2^c \stackrel{Q3}{\in} \text{DCFL} \label{Q41}\\
	&L_1 \cup L_2 = (L_1^c\cap L_2^c)^c \stackrel{(\ref{Q41})}{\in} \text{DCFL} \label{Q42}
\end{align}
The "element of"\text{ } statement of (\ref{Q41}) is due to the result of question 3 applied to an intersection of a regular language and a deterministic context free language, since both of these categories are closed under complements. In (\ref{Q42}) we once again use the closure property of DCFL's. \qed
\end{document}