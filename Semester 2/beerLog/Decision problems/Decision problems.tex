\documentclass{article}
\usepackage{amssymb}
\usepackage[a4paper]{geometry}
\usepackage[utf8]{inputenc}
\usepackage[T1]{fontenc} 					% 
\usepackage{lmodern} 						% Skrifttype
\usepackage[danish]{babel}
\usepackage{amsmath,amssymb,bm,mathtools,amsthm}	% Matematik pakker
\usepackage{fancyhdr,lastpage}
\usepackage{subfiles}
\usepackage{graphicx,float}
\usepackage{graphicx}
\usepackage{SASnRdisplay}
\usepackage{mathrsfs}
\setcounter{secnumdepth}{3} 				% No numbering
\fancypagestyle{plain}{
	\fancyhf{}								% Clear header/footer
	\renewcommand{\headrulewidth}{0pt}
	\fancyfoot[C]{Page \thepage \hspace{1pt} of \pageref{LastPage}}
}
\pagestyle{plain}
\usepackage{graphicx}
\usepackage{caption}
\usepackage{subcaption}
\usepackage{appendix}
\usepackage{dsfont}
\usepackage{color}
\usepackage{multicol}
\usepackage{enumerate}
\usepackage{setspace}
\newtheorem{theorem}{Sætning}[section] 
\newtheorem{corollary}[theorem]{Korollar}
\newtheorem{lemma}[theorem]{Lemma}
\newtheorem{proposition}[theorem]{Proposition}
\newtheorem{definition}[theorem]{Definition}
\theoremstyle{remark}
\newtheorem{obs}[theorem]{Observation}
\newtheorem{remark}[theorem]{Bemærkning}
\newcommand*{\LargerCdot}{\raisebox{-0.25ex}{\scalebox{1.4}{$\cdot$}}}

\addto\captionsenglish{\renewcommand{\figurename}{Figur}}
\DeclarePairedDelimiter\abs{\lvert}{\rvert}
\numberwithin{equation}{section}

\begin{document}
	\author{Jens Kristian Refsgaard Nielsen}
	\title{List of decision problems}
	\date{\today}
	\maketitle
\section{The theory}
A decision problem is something that takes an input and asks a question, we wish to determine whether it is possible to make a general algorithm that can answer this question.\\
\subsection{Reduces to, $P_1\leq P_2$}
We say a decision problem $P_1=(I_1,Q_1)$ reduces to another problem $P_2=(I_2,Q_2)$ iff $\exists F:I_1\rightarrow I_2$ such that $\forall i \in I_1 : i $ is a yes instance of $Q_1$ iff $F(i)$ is a yes instance of $Q_2$.
\subsection{Reduction theorem}
Given two problems and a reduction $P_1\leq P_2$ we get two very important results:\\
\textbf{Decideable} if $P_2$ is decideable then so is $P_1$,
\\\indent\textit{"if you live inside something solveable, then you too are solveable"}\\
\textbf{Undecideable} if $P_1$ is undecideable then so is $P_2$,
\\\indent\textit{"if something unsolveable lives inside of you, there is no salvation in more data"}\\
\subsection{Rice theorem}
\subsubsection{Language property}
\subsubsection{Examples}
\section{The list}
In order to decide whether something is decideable or not, it is very important to have a good list of results to use for our deductions. What follows are lists of problems handled in the lectures and exercise classes of "Beregnelighed \& Logik" at Aarhus university in the spring of 2019.\\
Each will be of the form: Name I: input Q: question\\
Notation:
\begin{align*}
	\text{TM }&=\text{ set of turing machines}\\
	\text{CFG }&=\text{ set of context free grammars}\\
	\text{REG }&=\text{ set of regular languages}
\end{align*}
\subsection{Decideable problems}
\begin{align}
	&\text{} &&\text{I:} &&&\text{Q:}\\
	&\text{} &&\text{I:} &&&\text{Q:}
\end{align}
\subsection{Undecideable problems}
\begin{align}
	&\text{Halting} &&\text{I: T$\in$TM,}w\in\Sigma_\text{T} \text{,}&&&\text{Q: does T halt on }w?\\
	&\text{} &&\text{I:} &&&\text{Q:}
\end{align}
\end{document}