\documentclass{article}
\usepackage{amssymb}
\usepackage[a4paper]{geometry}
\usepackage[utf8]{inputenc}
\usepackage[T1]{fontenc} 					% 
\usepackage{lmodern} 						% Skrifttype
\usepackage[danish]{babel}
\usepackage{amsmath,amssymb,bm,mathtools,amsthm}	% Matematik pakker
\usepackage{fancyhdr,lastpage}
\usepackage{subfiles}
\usepackage{graphicx,float}
\usepackage{graphicx}
\usepackage{SASnRdisplay}
\usepackage{mathrsfs}
\setcounter{secnumdepth}{2} 				% No numbering
\fancypagestyle{plain}{
	\fancyhf{}								% Clear header/footer
	\renewcommand{\headrulewidth}{0pt}
	\fancyfoot[C]{Page \thepage \hspace{1pt} of \pageref{LastPage}}
}
\pagestyle{plain}
\usepackage{graphicx}
\usepackage{caption}
\usepackage{subcaption}
\usepackage{appendix}
\usepackage{dsfont}
\usepackage{color}
\usepackage{multicol}
\usepackage{enumerate}
\usepackage{setspace}
\newtheorem{theorem}{Sætning}[section]
\newtheorem{corollary}[theorem]{Korollar}
\newtheorem{lemma}[theorem]{Lemma}
\newtheorem{proposition}[theorem]{Proposition}
\newtheorem{definition}[theorem]{Definition}
\theoremstyle{remark}
\newtheorem{obs}[theorem]{Observation}
\newtheorem{remark}[theorem]{Bemærkning}
\newcommand*{\LargerCdot}{\raisebox{-0.25ex}{\scalebox{1.4}{$\cdot$}}}

\addto\captionsenglish{\renewcommand{\figurename}{Figur}}
\DeclarePairedDelimiter\abs{\lvert}{\rvert}
\numberwithin{equation}{section}

\begin{document}
	\author{Morten Fausing \& Jens Kristian Refsgaard Nielsen \& Thomas Vinther }
	\title{Assignment G2}
	\date{\today}
	\maketitle
	\section{}
	Given a number of regular expressions $E_1,\dots,E_n$ we wish to compute the intersection as a minimal DFA
	\begin{align*}
	\bigcap_{i=1}^n E_i
	\end{align*}
	To do this we apply De Morgans law on complements of sets, where $A^{c}:=A^{'}=X\setminus A=\{x\in X \mid x\notin A \}$
	\begin{align*} 
	(A\cap B)^{c} &\stackrel{\text{De Morgan}}{=} A^{c}\cup B^{c} \Rightarrow\\
	A\cap B &= ((A\cap B)^c)^c = (A^{c}\cup B^{c})^c \Rightarrow\\
	\left(\bigcup_{i=1}^n E_i^c\right)^c &= \bigcap_{i=1}^n (E_i^c)^c = \bigcap_{i=1}^n E_i
	\end{align*}
	With this result in hand we construct the following procedure:
	\begin{align}
	E_1, \dots, E_n &\xrightarrow{\text{Kleene, (JM) Theorem 3.25}} N_1,\dots, N_n \in \text{NFA}\\
	N_1,\dots, N_n &\xrightarrow{\text{Subset construction, (JM) Theorem 3.18}} D_1,\dots, D_n \in \text{DFA}\\
	D_1,\dots, D_n &\xrightarrow{\text{Complement}} D_1^{c}, \dots D_n^{c} \in \text{DFA}  \\
	D_1^{c}, \dots D_n^{c} &\xrightarrow{\Lambda \text{  Union}}\bigcup_{i=1}^n D_i^{c} = N \in \text{NFA}\\
	N &\xrightarrow{\text{Subset construction, (JM) Theorem 3.18}} D \in\text{DFA}\\
	D &\xrightarrow{\text{Complement}} D^c \in\text{DFA}\\
	D^c &\xrightarrow{\text{Minimisation, (JM) algorithm 2.40}} M \in \text{DFA}
	\end{align} 
	\textbf{Explanation of procedure}
	In step (1.1) we use the known operation, of composition from a regular expression to an NFA by using Kleenes theorem. In (1.2) and (1.5) we use the constructive Theorem 3.18, which states that for any NFA there exsists a DFA, accepting the same language. In (1.3) and (1.6) we wish to compute the DFA that accepts complement language, this is simply achieved by
	\begin{align*}
	M &= (Q,\Sigma,q_0,A,\delta) \Rightarrow\\
	M^c &= (Q,\Sigma,q_0,Q\setminus A,\delta)
	\end{align*}
	Note however that this procedure fails on incomplete DFA, and as such these steps should be interpreted as fixing incompleteness before finding the complement.
	In (1.4) we union all the DFA's using a $\Lambda$-step, as seen in lecture 2 slide 27, thereby not using the product automaton, which we cannot do in JFLAP and get a NFA. 
	In (1.7) we apply algorithm (JM) 2.40 to achieve the minimized DFA we seek.
	\section{}
	\subsection*{Step (1.1)}
	Expression 1: At least one letter: $(a+b+0+1)^*(a+b)(a+b+0+1)^*$ as a NFA\\
	\includegraphics[width=\textwidth]{E1_NFA.jpg}
	
	\newpage Expression 2:At least two digits $(a+b)^*(0+1)(a+b)^*(0+1)(a+b+0+1)^*$ as a NFA\\
	\includegraphics[width=\textwidth]{E2_NFA.jpg}
	
	Expression 3: No three 0’s in a row: $((\Lambda+0+00)(a+b+1))^*
	(\Lambda+0+00)$ as a NFA\\
	\includegraphics[width=\textwidth]{E3_NFA.jpg}\\
	\newpage
	\subsection*{Step (1.2)}
	\noindent Expression 1 as a DFA\\
	\includegraphics[width=\textwidth]{E1_DFA.jpg}\\
	\noindent Expression 2 as a DFA\\
	\includegraphics[width=\textwidth]{E2_DFA.jpg}\\
	\noindent Expression 3 as a DFA, with trap state\\
	\includegraphics[width=\textwidth]{E3_DFA.jpg}\\
	\subsection*{Step (1.3)}
	\noindent Expression 1 DFA complement\\
	\includegraphics[width=0.9\columnwidth]{E1_cDFA.jpg}\\
	\noindent Expression 2 DFA complement\\
	\includegraphics[width=0.9\columnwidth]{E2_cDFA.jpg}\\
	\noindent Expression 3 DFA complement\\
	\includegraphics[width=0.9\columnwidth]{E3_cDFA.jpg}\\
	\newpage\noindent It is unneccesary to minimize at this stage, but for the sake of readability we have chosen to do so.\\ Expression 1 DFA complement minimized\\
	\includegraphics[width=\textwidth]{E1_mcDFA.jpg}\\
	\noindent Expression 2 DFA complement minimized\\
	\includegraphics[width=\textwidth]{E2_mcDFA.jpg}\\
	\noindent Expression 3 DFA complement minimized\\
	\includegraphics[width=\textwidth]{E3_mcDFA.jpg}\\
	\subsection*{Step (1.4)}j
	Combination of the minimized DFA's via $\Lambda$ step, new state 9 added.\\
	\includegraphics[width=\textwidth]{E_mlambdaStep.jpg}\\
	\subsection*{Step (1.5)}
	After converting the NFA of step (1.4) to a DFA\\
	\includegraphics[width=\columnwidth]{E_DFATV.jpg}\\
	\subsection*{Step (1.6)}
	After taking the complement of the DFA: \\
	\includegraphics[width=\columnwidth]{E_cDFATV.jpg}\\
	\subsection*{Step (1.7) the final result}
	Minimizing the DFA, note that it was already minimal \\
	\includegraphics[width=\columnwidth]{E_MincDFATV.jpg}\\
\end{document}