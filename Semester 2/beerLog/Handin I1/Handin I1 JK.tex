\documentclass{article}
\usepackage{amssymb}
\usepackage[a4paper]{geometry}
\usepackage[utf8]{inputenc}
\usepackage[T1]{fontenc} 					% 
\usepackage{lmodern} 						% Skrifttype
\usepackage[danish]{babel}
\usepackage{amsmath,amssymb,bm,mathtools,amsthm}	% Matematik pakker
\usepackage{fancyhdr,lastpage}
\usepackage{subfiles}
\usepackage{graphicx,float}
\usepackage{graphicx}
\usepackage{SASnRdisplay}
\usepackage{mathrsfs}
\setcounter{secnumdepth}{2} 				% No numbering
\fancypagestyle{plain}{
	\fancyhf{}								% Clear header/footer
	\renewcommand{\headrulewidth}{0pt}
	\fancyfoot[C]{Side \thepage \hspace{1pt} of \pageref{LastPage}}
}
\pagestyle{plain}
\usepackage{graphicx}
\usepackage{caption}
\usepackage{subcaption}
\usepackage{appendix}
\usepackage{dsfont}
\usepackage{color}
\usepackage{multicol}
\usepackage{enumerate}
\usepackage{setspace}
\newtheorem{theorem}{Sætning}[section]
\newtheorem{corollary}[theorem]{Korollar}
\newtheorem{lemma}[theorem]{Lemma}
\newtheorem{proposition}[theorem]{Proposition}
\newtheorem{definition}[theorem]{Definition}
\theoremstyle{remark}
\newtheorem{obs}[theorem]{Observation}
\newtheorem{remark}[theorem]{Bemærkning}
\newcommand*{\LargerCdot}{\raisebox{-0.25ex}{\scalebox{1.4}{$\cdot$}}}

\addto\captionsenglish{\renewcommand{\figurename}{Figur}}
\DeclarePairedDelimiter\abs{\lvert}{\rvert}
\numberwithin{equation}{section}

\begin{document}
	\author{Jens Kristian Refsgaard Nielsen}
	\title{Assignment I1}
	\date{21-02-2019}
	\maketitle
\section*{1.}
We wish to show that $L_1 := \{ a^i b^j a^j\mid 0\leq i , 0\leq j \}$ is non regular, we proceed by applying the pumping lemma.\\Let $k\geq 0$ be given, choose $z = a^p b^q a^q $ for some $p,q\geq0$ such that $|z|> k$ i.e. $p+2q>k$, let $u,v,w \in\Sigma^*$ be given such that $z=uvw$ and $|uv|< k$ and $|v| > 0 $ we then get a few cases of how $u,v,w$ could look:
\\\textbf{Case 1:} $u = a^l$, $v = a^j$, $w = a^{p-j-l}b^qa^q$ for $l\geq 0$ and $p\geq j>0$ such that $p-l-j\geq 0$.
\\\textbf{Case 2:} $u = a^{p-l}$, $v = a^{l} b^j$, $w = b^{q-j}a^q$ for $p\geq l\geq0$ and $j>0$ since if $j=0$ we are in case 1.
\\\textbf{Case 3:} $u = a^pb^j$, $v = b^l$, $w = b^{q-j-l}a^q$ for $l>0$ and $j\geq 0$ such that $q-l-j\geq 0$.
\\\textbf{Case 4:} $u = a^pb^{q-j}$, $v = b^{j}a^l$, $w = a^{q-l}$ for $l>0$ and $j\geq 0$ and $|uv|=p+q+l < k$.
\\Is it overkill to go through all these cases? absolutely.
\\\textbf{Case 1:} Consider
\begin{align*}
	uv^iw &= a^l a^{ij} a^{p-j-l}b^qa^q = a^{l+ij+p-j-l}b^qa^q 
	\\&= a^{(i-1)j+p}b^qa^q \in L_1	\iff (i-1)j+p \in \mathbb{N}
\end{align*}
Now we want to find an $i\geq 0$ such that $uv^iw\notin L_1$, but as shown above we cannot choose such an $i$ since $j$ is positive and smaller than $p$, and as such we are forced to choose $p=0$ to avoid case 1 altogether.
\\After forcing $p=0$ case 2 becomes absorbed into case 3, and becomes:
\\\textbf{Case 3':} $u = b^j$, $v = b^l$, $w = b^{q-j-l}a^q$ for $l>0$ and $j\geq 0$ such that $q-l-j\geq 0$. Consider
\begin{align*}
	uv^iw &= b^j b^{il} b^{q-j-l} a^q = b^{q+(i-1)l} a^q \\
	uv^iw &\in L_1 \iff q+(i-1)l = q \iff (i-1)l = 0
\end{align*}
As in case 1, we wish $uv^iw\notin L_1$ so we can choose $i\neq 1$ since $l>0$.
\\\textbf{Case 4:} After forcing $p=0$ case 4 now has $u = b^{q-j}$ and $q+l<k$, consider
\begin{align*}
	uv^iw &= b^{q-j} (b^{j}a^{l})^i a^{q-l}
\end{align*}
This looks very unlikely to be in $L_1$, but if we subdivide into cases of $j$ being zero or not we get
\begin{align*}
	j = 0 \Rightarrow& uv^iw = b^{q} a^{il} a^{q-l} = b^qa^{q+(i-1)l} \in L_1 \iff \\
					 & q = q+(i-1)l \iff i=1\\
	j > 0 \Rightarrow& uv^iw \in L_1 \Rightarrow i = 0 \vee i = 1 
\end{align*}
The $j>0$ case clearly breaks down if $i\geq 2$ because we cant have any $(abab)$ in our language, so we simply choose $i>1$ and this case has been taken care of.\\
In conclusion we can choose $z = b^qa^q$ where $2q>k$ and the pumping lemma will show that $L_1$ is non regular. \qed  
\section*{2.}
Consider $L_2 := \{ a^i b^jb^j \mid i\geq 0, j\geq 0 \} = \{ a^i (bb)^{j} \mid i\geq 0, j\geq 0 \} = \mathcal{L}(a^*(bb)^*)$ so $L_2$ is regular.\\
After spending (too) many hours trying to make the pumping lemma work on this set, I face palmed so hard that I am unable to formulate a coherent thought about how the above is non trivial.
\section*{3.}
Let $L_3$ be an arbitrary finite language, and $L_4$ be an arbitrary non-regular language. We wish to prove that $L_4\setminus L_3\in$NREG. Recall that for any sets we have
\begin{align*}
A = A\setminus B \cup (A \cap B)
\end{align*}
We take A, remove the bits of B in A and add those bits back in. Now if we assume for contradiction that $L_4\setminus L_3$ is regular, recall that REG is closed under union and intersection, and that all finite languages are regular. Note that $L_4 \cap L_3$ is finite since $L_3$ is, consider then
\begin{align*}
	L_4 = L_4\setminus L_3 \cup (L_4 \cap L_3)
\end{align*}
This is a union of regular languages and a contradiction, since $L_4$ is non regular. So $L_4\setminus L_3$ is non regular.
\end{document}