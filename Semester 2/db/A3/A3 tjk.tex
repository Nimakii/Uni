\documentclass{article}
\usepackage{amssymb}
\usepackage[a4paper]{geometry}
\usepackage[utf8]{inputenc}
\usepackage[T1]{fontenc} 					% 
\usepackage{lmodern} 						% Skrifttype
\usepackage[danish]{babel}
\usepackage{amsmath,amssymb,bm,mathtools,amsthm}	% Matematik pakker
\usepackage{fancyhdr,lastpage}
\usepackage{subfiles}
\usepackage{graphicx,float}
\usepackage{graphicx}
\usepackage{SASnRdisplay}
\usepackage{mathrsfs}
\setcounter{secnumdepth}{2} 				% No numbering
\fancypagestyle{plain}{
	\fancyhf{}								% Clear header/footer
	\renewcommand{\headrulewidth}{0pt}
	\fancyfoot[C]{Page \thepage \hspace{1pt} of \pageref{LastPage}}
}
\pagestyle{plain}
\usepackage{graphicx}
\usepackage{caption}
\usepackage{subcaption}
\usepackage{appendix}
\usepackage{dsfont}
\usepackage{color}
\usepackage{multicol}
\usepackage{enumerate}
\usepackage{setspace}
\usepackage{xcolor}
\newtheorem{theorem}{Sætning}[section]
\newtheorem{corollary}[theorem]{Korollar}
\newtheorem{lemma}[theorem]{Lemma}
\newtheorem{proposition}[theorem]{Proposition}
\newtheorem{definition}[theorem]{Definition}
\theoremstyle{remark}
\newtheorem{obs}[theorem]{Observation}
\newtheorem{remark}[theorem]{Bemærkning}
\newcommand*{\LargerCdot}{\raisebox{-0.25ex}{\scalebox{1.4}{$\cdot$}}}

\addto\captionsenglish{\renewcommand{\figurename}{Figur}}
\DeclarePairedDelimiter\abs{\lvert}{\rvert}
\numberwithin{equation}{section}

\begin{document}
	\author{Jens Kristian R. Nielsen \& Thomas D. Vinther }
	\title{Assignment G1}
	\date{\today}
	\maketitle
	Consider the relation R, which has attributes that hold schedules of courses and sections at
	a university (shorthand notation written in parenthesis): R = {Course Number (C), Section
		Number (SN), Offering Department (D), Course Type (CT), Credit Hours (CH), Course
		Level (CL), Instructor (I), Semester (S), Year (Y), Day and Hours (DH), Room Number
		(R), Number of Students Enrolled (NS)}.
	
	
	\section{Compute the minimal cover of \textbf{F}}
	%% page 540
	\begin{align*}
	F = \{\text{ } \{C\} &\rightarrow \{D,CT,CH,CL,I \}\\
	\{C,SN,S,Y\} &\rightarrow \{R,DH,NS,I \}\\
	\{R,DH,S,Y\} &\rightarrow \{I,C,SN \}\\
	\{I\} &\rightarrow \{D \}\\
	\{CT\} &\rightarrow \{CH,CL\}\\
	\{D,CT\} &\rightarrow \{I \}\\
	\{I,C\} &\rightarrow \{D,CT \}\text{}\}
	\end{align*}
	We compute the minimal cover of F, using the technique described in the lectures. First we reduce right hand sides and get \\\textbf{Step 1 and 2} single right hand attributes and reducing left hand sides
	\begin{align*}
	F' = \{\text{ } 
	\{C\} &\rightarrow \color{blue}\{D \}\\
	\{C\} &\rightarrow \color{yellow}\{CT\}\\
	\{C\} &\rightarrow \{CH \}\\
	\{C\} &\rightarrow \{CL\}\\
	\{C\} &\rightarrow \color{red}\{I \}\\
	\{C,SN,S,Y\} &\rightarrow \{R\}\\
	\{C,SN,S,Y\} &\rightarrow \{DH\}\\
	\{C,SN,S,Y\} &\rightarrow \{NS\}\\
	\color{red}\{C,SN,S,Y\} &\rightarrow \{I\}\\
	\{R,DH,S,Y\} &\rightarrow \{I\}\\
	\{R,DH,S,Y\} &\rightarrow \{C\}\\
	\{R,DH,S,Y\} &\rightarrow \{SN \}\\
	\{I\} &\rightarrow \{D \}\\
	\{CT\} &\rightarrow \{CH\}\\
	\{CT\} &\rightarrow \{CL\}\\
	\{D,CT\} &\rightarrow \{I \}\\
	\color{yellow}\{I,C\} &\rightarrow \{CT\}\\
	\color{blue}\{I,C\} &\rightarrow \{D\}\text{}\}
	\end{align*}
	We remove the functional dependency with left hand side in a color, due to the FD with the right hand side in the same color.
	\begin{align*}
	F'' = \{\text{ } 
	\{C\} &\rightarrow \{D \}\\
	\{C\} &\rightarrow \{CT\}\\
	\{C\} &\rightarrow \{CH \}\\
	\{C\} &\rightarrow \{CL\}\\
	\{C\} &\rightarrow \{I \}\\
	\{C,SN,S,Y\} &\rightarrow \{R\}\\
	\{C,SN,S,Y\} &\rightarrow \{DH\}\\
	\{C,SN,S,Y\} &\rightarrow \{NS\}\\
	\{R,DH,S,Y\} &\rightarrow \{I\}\\
	\{R,DH,S,Y\} &\rightarrow \{C\}\\
	\{R,DH,S,Y\} &\rightarrow \{SN \}\\
	\{I\} &\rightarrow \{D \}\\
	\{CT\} &\rightarrow \{CH\}\\
	\{CT\} &\rightarrow \{CL\}\\
	\{D,CT\} &\rightarrow \{I\} \}
	\end{align*}
	For step 3 consider
	\begin{align*}
	\{ C\} \rightarrow \{ I \} \rightarrow \{ D \} 
	\stackrel{\text{transitivity} }{\Rightarrow} \{ C  \} &\rightarrow \{ D \} \\
	\{ C\} \rightarrow \{ CT \} \rightarrow \{ CL, CH \} 
	\stackrel{\text{transitivity} }{\Rightarrow} \{ C  \} &\rightarrow \{ CL, CH \}
	\end{align*}
	In conclusion $\{C\} \rightarrow \{D\}$ and $\{C\} \rightarrow \{CL\}$ and $\{C\} \rightarrow \{CH\}$ are redundant.
	Which leaves us with the functional dependencies
	\begin{align*}
	F''' = \{\text{ } 
	\{C\} &\rightarrow \{CT\}\\
	\{C\} &\rightarrow \{I \}\\
	\{C,SN,S,Y\} &\rightarrow \{R\}\\
	\{C,SN,S,Y\} &\rightarrow \{DH\}\\
	\{C,SN,S,Y\} &\rightarrow \{NS\}\\
	\{R,DH,S,Y\} &\rightarrow \{I\}\\
	\{R,DH,S,Y\} &\rightarrow \{C\}\\
	\{R,DH,S,Y\} &\rightarrow \{SN \}\\
	\{I\} &\rightarrow \{D \}\\
	\{CT\} &\rightarrow \{CH\}\\
	\{CT\} &\rightarrow \{CL\}\\
	\{D,CT\} &\rightarrow \{I\} \}
	\end{align*}
	\section{}
	\textbf{Based on your answer to (1), find the candidate keys of R based on F.}
	Note that $C$ determines $\{CT,I,D,CH,CL\}$ wrt. $F'''$ and the set that $C$ determines is all the "small" functional dependencies
	\textbf{Claim:} $\{ C,SN,S,Y\}$ and $\{ R, DH, S, Y \}$ are candidate keys.
	Show that they are irreducible page 541ish
	\section{}
	\textbf{Normalize R to a lossless BCNF decomposition with respect to F.}
	\section{}
	\textbf{Is your BCNF decomposition dependency-preserving? Why?}
	\section{}
	\textbf{Can you produce a dependency-preserving BCNF decomposition of R? Why?}
	\section{}
	\textbf{Synthesize a lossless and dependency-preserving decomposition of R in 3NF.
		Explain what foreign key constraints you will use to guarantee losslessness.}
	\section{}
	\textbf{Suggest a way to enforce all dependencies in F via integrity constraints in SQL DDL
		(e.g., PRIMARY KEY, FOREIGN KEY, UNIQUE). Specifically, go through each
		dependency in the minimal cover of F and explain how it is now enforced.}
	\section{}
	\textbf{Assuming you have answered Question 7 correctly, some dependency among
		attributes within a table, which we call child table, must be enforced using a foreign
		key constraint to a parent table (if not, you might consider this as a hint). Write a
		trigger that enforces that same dependency, in case of an insertion to the child table.}
	
\end{document}

%%\begin{align*}
%%     \{ C, SN, S, Y \} &= \{ C, SN, S, Y \} \cup \{S,Y\} \stackrel{\text{reflexsivity}}{\rightarrow} \{ R, DH \} \cup \{ S, Y \} \rightarrow \{ I,C \} \\
%%     \stackrel{\text{transitivity} }{\Rightarrow} \{ C, SN, S, Y \} &\rightarrow \{ I,C \}
%%\end{align*}